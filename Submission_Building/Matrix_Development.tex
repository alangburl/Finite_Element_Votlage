\documentclass{article}[12pt]
\usepackage{amsmath}
\usepackage{gensymb}
\usepackage{standalone}
\usepackage{verbatim}
\usepackage[utf8]{inputenc}
\usepackage{setspace}
\usepackage[a4paper,margin=1in,footskip=0.25in]{geometry}
\usepackage{graphicx}
\usepackage{mathptmx}
\usepackage{booktabs}
\usepackage{cite}
\usepackage[english]{babel}
\usepackage[utf8]{inputenc}

\begin{document}
\section{Development of Matrices and solution}
	Using input from the user interface, a matrix of dimension $nxm$ with $n$ and $m$ being the number specified in the $x$ and $y$ directions. The spaing and placement of the nodes is done using Python3.7 NumPy module. Nodes of the geometry are broken down into boundary conditions and field conditions. 
\subsection{Boundary Conditions}
	Boundary condition locations are taken form the user interface by means of the terminal geometry widget. In this widget, the geometric location and size of the terminal is entered as well as the potential applied across the terminal. At this point, the terminal is limited to having an uniform distribution across the entire surface. These inputs are passed as arguments to the boundary condition method of Voltage Calculation class in Calculation.py. The specified voltage is used as the boundary condition and is placed assigned to its respective location in the $b$ matrix. The remainder of the boundary conditions are determined by reflecting the adjacent node normal to the boundary condition. For a terminal with known potential on the right side of the geometry, an example is shown in (\ref{eq:right_boundary}).
		\begin{equation}
		\frac{\partial{^2V}}{\partial{x^2}}+\frac{\partial{^2V}}{\partial{y^2}} \approx \frac{\left(V_{x+1}-V_{xy}\right)-\left(V_x-V_{x+1}\right)}{\Delta x^2}+\frac{\left(V_{y-1}-V_{xy}\right)-\left(V_{xy}-V_{y+1}\right)}{\Delta y^2}=V_{ap}
		\label{eq:right_boundary}
	\end{equation}
Where $V_{xy}$ is the boundary node being evaluated, and $\Delta x$, $\Delta y$ are the distance between nodes in the $x$ and $y$ directions respectively. (\ref{eq:right_boundary}) is reduced to have a coefficient of one for the node in question for matrix solving, seen in (\ref{eq:reduced_boundary}).
	\begin{equation}
		V_{xy}-\frac{V_{x+1}\Delta y^2}{\Delta x^2+\Delta y^2}-\frac{V_{y-1}\Delta x^2}{\Delta  x^2+\Delta y^2}-\frac{V_{y+1}\Delta x^2}{\Delta  x^2+\Delta y^2}=\frac{-\rho \Delta  x^2 \Delta y^2}{2\epsilon\left(\Delta x^2+\Delta y^2\right)}
		\label{eq:reduced_boundary}
	\end{equation}
The remainder of the boundary conditions are solved using the same method with their respective nodes. 
\subsection{Field Conditions}
	After determination of the boundary conditions is complete, the field of the geometry is calculated. All nodes in the field are determined using (\ref{eq:field_values}).  The implementation of a space charge term, $\rho$, can be done in this process. 
	\begin{equation}
		V_{xy}-\frac{V_{x+1}\Delta y^2}{2\left(\Delta x^2+\Delta y^2\right)}-\frac{V_{x-1}\Delta y^2}{2\left(\Delta x^2+\Delta y^2\right)}-\frac{V_{y-1}\Delta x^2}{2\left(\Delta  x^2+\Delta y^2\right)}-\frac{V_{y+1}\Delta x^2}{2\left(\Delta  x^2+\Delta y^2\right)}=\frac{-\rho \Delta  x^2 \Delta y^2}{2\epsilon\left(\Delta x^2+\Delta y^2\right)}
		\label{eq:field_values}
	\end{equation}
\subsection{Calculation}
	After all nodes in the matrix have been accounted for, the A matrix will have ones on the diagonal with other constituents on the same row. Use of Python3.7 NumPy module is used to solve for the voltage vector. This vector is returned to the user interface by means of a contour plot.
\end{document}
