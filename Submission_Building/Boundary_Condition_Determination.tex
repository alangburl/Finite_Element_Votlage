\documentclass{article}[12pt]
\usepackage{amsmath}
\usepackage{gensymb}
\usepackage{standalone}
\usepackage{verbatim}
\usepackage[utf8]{inputenc}
\usepackage{setspace}
\usepackage[a4paper,margin=1in,footskip=0.25in]{geometry}
\usepackage{graphicx}
\usepackage{mathptmx}
\usepackage{booktabs}
\usepackage{cite}
\usepackage[english]{babel}
\usepackage[utf8]{inputenc}

\begin{document}
\section{Boundary Conditions}
	Boundary conditions are input to the system by means of the GUI interface, found in GUI.py. The boundary conditions are based on the input geometry and voltages associated with each terminal. These are handled by the boundary condition method in Calculation.py. Boundary conditions located on a terminal have a $\frac{\rho}{\epsilon}$ of the voltage applied to the terminal, which is placed in the b matrix for solving of the matrix. The remainder of the boundary conditions are determined by use of ?????. This mirrors the adjacent node on the same axis for finite approximation. For a terminal on the right side of the geometry, an example is shown in (\ref{eq:right_boundary}).
	\begin{equation}
		\frac{\left(V_{x+1}-V_{xy}\right)-\left(V_x-V_{x+1}\right)}{\Delta x^2}+\frac{\left(V_{y-11}-V_{xy}\right)-\left(V_{xy}-V_{y+1}\right)}{\Delta y^2}=\frac{\rho}{\epsilon}
		\label{eq:right_boundary}
	\end{equation}
Where $V_{xy}$ is the boundary node being evaluated, and $\Delta x$, $\Delta y$ are the distance between nodes in the $x$ and $y$ directions respectively. (\ref{eq:right_boundary}) is reduced to have a coefficient of one for the node in question for matrix solving, seen in (\ref{eq:reduced_boundary}).
	\begin{equation}
		V_{xy}-\frac{V_{x+1}\Delta y^2}{\Delta x^2+\Delta y^2}-\frac{V_{y-1}\Delta x^2}{\Delta  x^2+\Delta y^2}-\frac{V_{y+1}\Delta x^2}{\Delta  x^2+\Delta y^2}=\frac{-\rho \Delta  x^2 \Delta y^2}{2\epsilon\left(\Delta x^2+\Delta y^2\right)}
		\label{eq:reduced_boundary}
	\end{equation}
\end{document}